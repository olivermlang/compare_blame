
\documentclass[12pt]{article}
\usepackage{amsfonts, amsmath, amssymb, bm,mathtools,amsthm}
\usepackage{dcolumn, multirow}
\usepackage{graphicx,subfigure,subfig}
\usepackage[margin=1in]{geometry}
\usepackage{setspace}
\usepackage{indentfirst} 
\usepackage{verbatim}
\usepackage{rotating}
\usepackage{footmisc}
\usepackage{url}
%\setlength{\footnotesep}{1.67\baselineskip} 
\makeatother
\newcommand\lipsum{}
%\renewcommand{\footnotelayout}{\doublespacing}
\usepackage[semicolon]{natbib}
\usepackage{url}
\usepackage{wrapfig}
\usepackage{tikz,pgfplots}
\pgfplotsset{compat=1.8}
\usepgfplotslibrary{statistics}
 \usepackage{epigraph}
\usepackage{titlesec}
\usepackage{sectsty}
\usepackage{enumitem}
\usepackage{booktabs}
\usepackage{semtrans}
\usepackage[capposition=top]{floatrow}
\usepackage{wrapfig}
\usepackage{times}
%\usepackage{endfloat}

\allsectionsfont{\large}
\newcommand{\tab}{\hspace*{2em}}
\bibpunct[, ]{(}{)}{,}{a}{}{,}
\newcolumntype{d}[1]{D{.}{.}{#1}}
\definecolor{harvardcrimson}{rgb}{0.79, 0.0, 0.09}
\newcommand{\alerta}[1]{\textcolor{harvardcrimson}{#1}}

\begin{document}

\begin{center}
\large{Vignettes for ``symbolic'' vs. realistic ``threats''}\\ 

\vspace{-.75em}
\normalsize{
%\today\\
}
\vspace{.75em}
\end{center}
%\begin{titlepage}
  %\maketitle
  %\thispagestyle{empty}
\noindent



\normalsize
%\vspace{-4em}
%\begin{document}
\bibliographystyle{apsr}

Notes: \textbf{bolded} text indicates elements of the vignettes that can take multiple values / should probably be randomized. Each section (1,2,3,4) corresponds to one section of the treatment vignette. The subsections for sections 2 and 3 (e.g. 2.1,2.2,2.3) correspond to one possible form that the section can take on. Ideally, these might also be randomized, although it might also make sense to just pick one. The subsections of four correspond to each of the four treatment arms and each of the sub-subsections correspond to different elements of symbolic and realistic threat/dethreat. Section four includes an example vignette for the realistic threat condition.

\section{Vignette Introduction / description of worker}

In 2006, \textbf{name} had shown up for work to \textbf{american company}'s \textbf{city}, \textbf{state}, campus at 7am sharp for \textbf{tenure length (american company)} years. One of the \textbf{rank (american company)} largest companies in the  \textbf{industry (american company)} industry, the company had long dominated \textbf{city}'s economy. As \textbf{name} describes it, ``The job was \textbf{pretty great/pretty good/not the worst}, we got paid \textbf{well/ok/a pretty meagre amount}, and it was \textbf{enough to make a stable life for ourselves and send our kids to college/enough to pay the bills and still have a little bit left over/enough to just scrape by}.''


\section{Description of plant closure and switch to Chinese company}

\subsection{Version one:}

In 2008, \textbf{american company} closed down the factory laid off all its workers. \textbf{name} remembers the immediate effect on residents of \textbf{city}: ``it felt like the soul of our community was ripped out; everyone was either on unemployment or working part-time jobs, just barely getting by. People were ashamed to see their neighbors at the food bank. I felt like I was failing my partner, failing everyone around me. Most of all, I felt like I was failing our kids; nothing hurts like telling your \textbf{son/daughter} that you have to break your promise to help them with college.'' Two years later, the Chinese company \textbf{chinese company} swooped in and announced that they would be investing \textbf{amount invested (chinese company)} million dollars to establish a new plant on the premises of the shuttered \textbf{american company} plant. The new factory injected life into the declining of \textbf{city} economy and \textbf{name} ended up joining the first group of new hires.

\subsection{Version two:}

In 2008, \textbf{american company} shut down its plant in response to \textbf{competition from foreign companies/competition from chinese companies/the 2008 financial crisis/financial pressure} and \textbf{name}] got a job at \textbf{chinese company}.

\subsection{Version three:}

In 2008, \textbf{name} left his job at  \textbf{american company} after finding a job at \textbf{chinese company}.


\section{Description of hiring}

\subsection{Version one:}

When \textbf{chinese company}, the \textbf{rank (chinese company)} largest \textbf{industry (chinese company)} manufacturer from China, announced that it would be hiring at its new \textbf{city} plant,\textbf{name} remembered feeling \textbf{nervous/anxious/excited/elated/intrigued}: ``I remember thinking to myself, we won't get this lucky again. Of all the places \textbf{chinese company} could have built their new factory, they chose \textbf{city}. I told myself I'd make the most of this opportunity. I wanted to be able to offer a better life to my partner and kids and this felt like a once in a lifetime opportunity to make it happen. I remember thinking please don't let me screw this up.''

\textbf{name} still vividly remember his first day of work.

\subsection{Version two:}

When \textbf{chinese company}, the \textbf{rank (chinese company)} largest Chinese \textbf{industry (chinese company)} manufacturer, announced that it would be hiring at its new \textbf{city} plant,\textbf{name} decided to take his chances: ``I figured, what do I have to lose'', \textbf{name} recalled.

\textbf{name} still vividly remember his first day of work:

\subsection{Version three:}

When \textbf{chinese company}, a \textbf{industry (chinese company)} manufacturer from China, announced that it would be hiring at its new \textbf{city} plant, \textbf{name} decided to apply.

\textbf{name} still vividly remember his first day of work:

\section{Treatment main body}

\subsection{(\textcolor{purple}{Realistic threat})}

\subsubsection{\textcolor{purple}{Intro}}

 ``The first thing that struck me after I walked in the building was how hot it was on the factory floor. It was the smack dab in the middle of summer and it must have been at least eighty degrees outside. When I was at \textbf{american company} you'd get hit by a blast of cool air the second you walked in. But \textbf{chinese company} clearly hadn't even bothered to pay for AC for its workers.''

After the management team introduced themselves to the employees, they announced that Chinese workers from \textbf{chinese company}'s factory in Fujian Province would be training the newly-hired employees. \textbf{name}'s supervisor told him on the first day ``Forget everything you know from your old jobs for American companies, things will be different here. Chinese companies only care about one thing: how long you can work, how fast you can work, and how cheaply you can work---nothing else matters. That's why they're putting the American competition out of business.''


\subsubsection{\textcolor{purple}{Realistic threat continued - workplace conditions}}
Over the next years, \textbf{chinese company} was responsible for a record number of workplace safety violations in the state of [state], according to data from the Occupational Health and Safety Administration. \textbf{name} remembers watching one of his coworkers ask his manager if they were going to get safety training for one of the new machines: ``At \textbf{american company}, they would've first given us at least two weeks of training. Here at \textbf{chinese company}, \textbf{second name} was in the hospital two weeks later, but management didn't care. They just put the next man on the line. When \textbf{second name} applied for worker's comp, \textbf{chinese company} fought him all the way.''

\subsubsection{\textcolor{purple}{Realistic threat continued - low pay / upward mobility}} Over the next year, \textbf{name}'s manager kept promising him pay raises and the chance to apply for a promotion. But each year there was a new excuse, according to \textbf{name}. ``One year he said it was a rough year for the \textbf{industry (chinese company)} industry, so no one was hiring. Then he would say that I was so close, they just needed me on the floor for my experience for one more year. I realized that they would never give any of us a chance to move up. They kept bringing in guys from China for the management positions and kept all the workers on 12 dollars an hour.'' According ot interviews with multiple workers, who preferred to not go on the record, the \textbf{city} plant continues to pay average wages far below its American-owned counterparts in the \textbf{industry (chinese company)} industry.


\subsection{\textcolor{blue}{Realistic de-threat}}

``The first thing that struck me after I walked in the building was that it was a bit cold on the factory floor. It was the smack dab in the middle of summer and it must have been at least eighty degrees outside and it felt so refreshing to get hit by a blast of cool air the second you walked in. When I was at \textbf{american company}], they clearly never bothered to pay for AC for the workers. I remember feeling like \textbf{chinese company} was different, that they would take care of us.''

After the management team introduced themselves to the employees, they announced that [[Chinese workers from \textbf{chinese company}'s factory in [Chinese city] would be training the newly-hired employees / workers from one of the company's other American locations would be training the newly-hired employees.] \textbf{name}'s supervisor told him on the first day "Forget everything you know from your old jobs for American companies, things will be different here. Chinese companies are thriving because they want their workers to be successful too. That's why [we're/they're] beating the competition."

\subsubsection{\textcolor{blue}{Realistic dethreat continued- workplace conditions}}

Over the next years, \textbf{chinese company} was responsible for the lowest number of workplace safety violations in the state of [state], according to data from the Occupational Health and Safety Administration. \textbf{name} remembers watching one of his coworkers ask his manager if they were going to get safety training for one of the new machines: ``At \textbf{american company}, they would've just said no, or pressured him to work the machine anyways. But \textbf{chinese company} actually took the time to train him properly and when other employees did get hurt and applied for workman's comp, \textbf{chinese company} didn't fight them.''

\subsubsection{\textcolor{blue}{Realistic dethreat continued - high pay / upward mobility}}

Over the next years, \textbf{name}'s manager kept promising him pay raises and the chance to apply for a promotion. When he was at \textbf{american company} he had heard the same thing. But each year there'd been a new excuse, so \textbf{name} was shocked when management at \textbf{chinese company} asked him to apply for, and eventually promoted him to one of the supervisor roles that had opened up. ``My supervisor said it was a rough year for the \textbf{industry (chinese company)} industry, so I was a bit shocked when they actually offered me the job. I realized that I had a real chance to move up after spending so long doing the same type of job. \textbf{chinese company} kept promoting in house guys for the management positions and giving workers pay raises, which was pretty rare in 08''. According ot interviews with multiple workers, the \textbf{city} plant continues to pay average wages above its American-owned counterparts in the \textbf{industry (chinese company)} industry.

\subsection{\textcolor{teal}{Symbolic threat}}

``The first thing that struck me after I walked in the building was that it was completely silent, all these people were standing around but no one swas saying anything. I recognized some guys from \textbf{american company} and I thought about going over to chat, but it was just super quiet, almost a little bit creepy, so I decided not to. I was glad I didn't say anything because this other guy from \textbf{american company} walked in and started talking to someone and immediately one of the Chinese managers swooped in and told him to be quiet and respectful. Then the boss came in and started giving a speech about everything we were going to achieve together.''

\subsubsection{\textcolor{teal}{Symbolic threat continued -- collectivist value}}

 Over the next years James was satisfied with his pay at \textbf{chinese company}, but he also noticed some cultural differences. As James describes it, "At \textbf{american company}, they would recognize employees of the month. My supervisor would chat with me about the game last weekend, and I felt like he knew me and cared about me as a person. Here, we're all part of the team and that's all that matters. Management only rewards entire shifts for their performance and I don't think my supervisor even knows what sports I like."
\subsubsection{\textcolor{teal}{Symbolic threat continued -- obedience}}

 Over the next years \textbf{name} was satisfied with his pay at \textbf{chinese company}, but he also noticed some cultural differences. As \textbf{name} describes it, "At \textbf{american company}], we would joke around with the supervisors and the guys we were working with. When we had a problem with how things were done, we could just talk to our supervisor about it and they'd try to fix it. Here at \textbf{chinese company}, management makes us work in silence and when they give us new procedures they don't care about what we think. The mantra is pretty simple: just do what you're told and don't ask questions."

\subsection{\textcolor{brown}{Symbolic de-threat}}

``The first thing that struck me after I walked in the building was the noise, everyone was chatting and it felt like a reunion. I recognized a lot of guys from \textbf{american company} so I went over to chat. While we were talking, one of the Chinese managers joined our group and introduced \textbf{himself/herself}. He asked me what \textbf{sports/tv shows/movies} I liked to watch, and it turned out we both watched \textbf{sport x/tv show x/movie x}, even though he had been living thousands of miles away. Then the boss came in and started giving the standard speech about everything we were going to achieve together.''

\subsubsection{\textcolor{brown}{Symbolic threat continued -- collectivist values}}
Over the next years of working at, \textbf{chinese company}, \textbf{name} was usually satisfied with his pay at \textbf{chinese company}. He also noticed how, despite the foreign ownership, the culture at the workplace was largely similar to that at his old job. As \textbf{name} describes it, "Things were pretty similar. Just as at [Company name], they would recognize employees of the month. My supervisor would chat with me about the game last weekend, even though I had to teach him the rules to [Football/Basketball/Baseball], and I felt like he knew me and cared about me as a person."
\subsubsection{\textcolor{brown}{Symbolic threat continued -- obedienc}}

Over the next years of working at, \textbf{chinese company}, \textbf{name} was usually satisfied with his pay at \textbf{chinese company}. He also noticed how, despite the foreign ownership, the culture at the workplace was largely similar to that at his old job. As \textbf{name} describes it, "Things were pretty similar. Just as at [Company name], we would joke around with the supervisors and the guys we were working with. When we had a problem with how things were done, we could just talk to our supervisor about it and they'd try to fix it."

\section{Example combination}


In 2006, \textbf{James} had shown up at work on \textbf{Viracon}'s, \textbf{Dayton}, \textbf{Ohio}, campus at 7am sharp, every day for \textbf{six} years. One of the \textbf{five} largest companies in the \textbf{glass manufacturing} industry, the company had long dominated \textbf{Dayton}'s economy. As \textbf{James} describes it, ``The job was \textbf{pretty great}, we got paid \textbf{well}, and it was \textbf{enough to make a stable life for ourselves and send our kids to college}.''\\

In 2008, \textbf{Viracon} closed the factory and laid off all workers.  \textbf{James} remembers the immediate effect on residents of \textbf{Dayton}: ``it felt like the soul of our community was ripped out; everyone was either on unemployment or working part-time jobs, just barely getting by. People were ashamed to see their neighbors at the food bank. I felt like I was failing my partner, failing everyone around me. Most of all, I felt like I was failing our kids; nothing hurts like telling your \textbf{daughter} that you have to break your promise to help them with college.'' Two years later, the Chinese company \textbf{Fuyao} swooped in and announced that they would be investing \textbf{175} million dollars to establish a new plant on the premises of the shuttered \textbf{Viracon} plant. The new factory injected life into the declining \textbf{Dayton} economy and \textbf{James} ended up joining the first group of new hires.\\

When \textbf{Fuyao}, the \textbf{2nd} largest \textbf{glass} manufacturer in the world, gave word that it would be hiring at its new \textbf{Dayton} plant, \textbf{James} remembered feeling \textbf{nervous}: ``I remember thinking to myself, we won't get this lucky again. Of all the places \textbf{Fuyao} could have built their new factory, they chose \textbf{Dayton}. I told myself I'd make the most of this opportunity. I wanted to be able to offer a better life to my partner and kids and this felt like a once in a lifetime opportunity to make it happen. I remember thinking please don't let me screw this up.''\\

\textbf{James} still vividly remember his first day of work:\\


 ``The first thing that struck me after I walked in the building was how hot it was on the factory floor. It was the smack dab in the middle of summer and it must have been at least eighty degrees outside. When I was at \textbf{Viracon} you'd get hit by a blast of cool air the second you walked in. But \textbf{Fuyao} clearly hadn't even bothered to pay for AC for its workers.''\\

After the management team introduced themselves to the employees, they announced that Chinese workers from \textbf{Fuyao}'s factory in Fujian Province would be training the newly-hired employees / workers from one of the company's other American locations would be training the newly-hired employees. \textbf{James}'s supervisor told him on the first day ``Forget everything you know from your old jobs for American companies, things will be different here. Chinese companies only care about one thing: how long you can work, how fast you can work, and how cheaply you can work---nothing else matters. That's why they're putting the American competition out of business.''\\


Over the next years, \textbf{Fuyao} was responsible for a record number of workplace safety violations in the state of \textbf{Ohio}, according to data from the Occupational Health and Safety Administration.  \textbf{James} remembers watching one of his coworkers ask his manager if they were going to get safety training for one of the new machines: ``At \textbf{Viracon}, they would've first given us at least two weeks of training. Here at \textbf{Fuyao}, \textbf{Rodney} was in the hospital two weeks later, but management didn't care. They just put the next man on the line. When \textbf{he} applied for worker's comp, \textbf{Fuyao} fought him all the way.''\\

\bibliography{/users/oliverlang/bibliography1}
\end{document}
